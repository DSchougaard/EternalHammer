\documentclass[a4paper,10pt]{article}
\usepackage[utf8]{inputenc}


\title{	
\textsc{Technical University of Denmark} \\ [25pt] 
\huge Project Agreement
}
\author{Daniel Schougaard \\ \textit{s103446}} 
\date{\normalsize\today}
\begin{document}
\maketitle 


\begin{tabular}{ | l | r | }
	Title (English)		&	Sharing Files Across Devices Using The Private Cloud				\\
	Title (Danish)		&	Fildeling På Tværs af Enheder Ved Brug af Den Private Sky			\\
	ECTS Points			&	32.5																\\
	Student Number		&	s103446																\\
	Start date 			&	4/1-2016 \textit{(dispensation has been given)}					\\
\end{tabular}



\section{Abstract of Project}
	Over the past few years, privacy on the internet has become a growing issue. As the cloud has become more and more popular for home users to use, so has it for hackers to attack. The recent LastPass security break \footnote{https://blog.lastpass.com/2015/06/lastpass-security-notice.html/} is a shining example of this.

	One way to avoid this, which I know from personal experience is being used, is to use for instance KeePass, and store the \verb=.kbdx= file in Dropbox. One downfall to this, is that you would have to rely on Dropbox to keep your -- granted encrypted -- password file safe.

	In this project, the feasibility of running a private file synchronization and password manager service on a device consuming as little power as possible will be investigated. The end result, would be a service somewhat similar to that of Dropbox, while also syncing your passwords, but kept in the Private Cloud: None of the files ever leave your devices, and are kept \emph{yours}.

	The goal of the project is to implement a feature rich server-application, handling storage, synchronization, password encryption, public share links and all around primary logic of the system. The implementation of the system, for this project, will be separated into two sub-systems: The server and a client. The server will be developed running on a Unix system, due to affordability and availability of low-powered devices, running a Unix core. A client will also be developed. Due to its mainstream use, the client will be developed for Windows.
\end{document}

